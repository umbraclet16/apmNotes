\documentclass[a4paper,10pt]{ctexart} %ctexart
%\usepackage{ctex}
%\usepackage{color}
%\usepackage{lineno} %打印行号的宏包,使用{linenumbers*}环境。
\usepackage[backref,hidelinks]{hyperref} %hidelinks去掉引用处的红框
\pagestyle{plain} %把页码从右上角移到页脚
%==========打印代码的设置===================
\usepackage{color}
\definecolor{gray}{rgb}{0.8,0.8,0.8}
\usepackage{listings}
\lstset{basicstyle=\small}
\lstset{numbers=left} \lstset{language=C++} \lstset{breaklines}
\lstset{extendedchars=false} \lstset{backgroundcolor=\color{gray}}
\lstset{keywordstyle=\color{blue}\bfseries} \lstset{frame=none}
\lstset{tabsize=4} \lstset{commentstyle=\color{red}}
\lstset{stringstyle=\emph}
%=======================================
\begin{document}
\begin{center}
\textbf{AP\_AHRS\_DCM.cpp}
\end{center}

\noindent \textbf{AP\_AHRS\_DCM::update(void)}\\
更新DCM矩阵。利用陀螺仪、加速度计和GPS的信息更新矩阵,矩阵被用于控制和导航。整个过程分为6步,即顺序调用以下6个函数:
\begin{description}
\item [\_ins.update()]:			读取IMU数据;
\item [matrix\_update()]:		使用陀螺仪的数据更新DCM矩阵;
\item [normalize()]:			将DCM矩阵标准化;
\item [drift\_correction()]:	修正漂移;
\item [check\_matrix()]:		检查矩阵中元素是否有坏值;
\item[euler\_angles()]:			计算欧拉角。
\end{description}
除\_ins.update()外,其余函数在此cpp中定义。\\
============================================

%111111111111111111111111111111111111
%111111111111111111111111111111111111
%111111111111111111111111111111111111
\noindent 1.\textbf{matrix\_update}

使用陀螺仪的数据更新DCM矩阵。算法涉及的核心概念是一个非线性微分方程,其将方向余弦的变化率和陀螺信号联系起来。此算法计算方向余弦时不会进行任何违背微分方程非线性的近似化处理。

由于旋转是不可交换的,因此不能简单地通过积分陀螺速率信号来得到角度。考虑一个向量r进行旋转的运动学方程:
\begin{equation}
\frac{d\mathbf{r}(t)}{dt}=\mathbf{\omega}(t)\times\mathbf{r}(t)
\end{equation}
此方程为非线性,$\omega$为我们想要积分的量。若初始条件已知,则对上式积分得到
\begin{equation}
\mathbf{r}(t)=\mathbf{r}(0)+\int\limits_{0}^{t}d\mathbf{\theta}(\tau)\times\mathbf{r}(\tau)
\end{equation}
\begin{displaymath}
d\mathbf{\theta}(\tau)=\mathbf{\omega}(\tau)d\tau
\end{displaymath}
算法的策略是将此方程应用于DCM矩阵的行或列,将它们作为旋转的向量对待。
注意到此方程中向量$\mathbf{r}$和向量$\mathbf{\omega}$需是在同一个参考系中测得的;我们希望跟踪机体轴在地系中的变化,而陀螺所测量是是以机体系为参考的,故方程不能直接应用。由于旋转具有对称性,在机体参考系下地轴的旋转可等价为惯性参考系下机体轴的旋转。因此,我们改为跟踪机体系下地系坐标轴的旋转。具体做法是改变陀螺信号(即方程中积分项)的符号为负,我们可以通过交换叉乘的顺序从而使符号再变回来:
\begin{equation}
\mathbf{r}_{earth}(t)=\mathbf{r}_{earth}(0)+\int\limits_{0}^{t}\mathbf{r}_{earth}(\tau)\times d\mathbf{\theta}(\tau)
\end{equation}
$\mathbf{r}_{earth}(t)$是在机体系下观测的地系任一坐标轴,即DCM矩阵的行向量。要按此方程进行计算,可以借助欧拉旋转定理和旋转向量的方法。
将方程变回微分方程的形式:
\begin{equation}\label{aaa}
\mathbf{r}_{earth}(t+dt)=\mathbf{r}_{earth}(t)+\mathbf{r}_{earth}(t)\times d\mathbf{\theta}(t)
\end{equation}
\begin{displaymath}
d\mathbf{\theta}(t)=\mathbf{\omega}(\tau)d\tau
\end{displaymath}
由于要针对陀螺的漂移进行修正,$\omega$具体包括两部分:
\begin{equation}
\mathbf{\omega}(t)=\mathbf{\omega}_{gyro}(t)+\mathbf{\omega}_{correction}(t)	%加粗没起作用
\end{equation}
$\omega_{gyro}(t)$是陀螺三轴上的数据,$\omega_{correction}(t)$是修正量(包含比例项和积分项,构成PI控制)。

对机体参考系下3个地系坐标轴按照方程(\ref{aaa})计算,可写成矩阵形式:
\begin{equation}
\mathbf{R}(t+dt)=\mathbf{R}(t)
\left( \begin{array}{ccc}
	    1      & -d\theta_z & d\theta_y  \\
	d\theta_z  &     1      & -d\theta_x \\
	-d\theta_y & d\theta_x  &     1
\end{array} \right)
\end{equation}
\begin{displaymath}
d\theta_x=\omega_xdt,\quad
d\theta_y=\omega_ydt,\quad
d\theta_z=\omega_zdt
\end{displaymath}
即旋转向量的方法,其已由Matrix.cpp中的rotate()函数实现。通过以小步长(如0.02s)重复进行以上矩阵乘法,即可实现DCM矩阵的实时更新。

\vspace{8pt}
\noindent 具体代码:
\begin{lstlisting}
    _omega=_ins.get_gyro()+_omega_I;
    _dcm_matrix.rotate((_omega+_omega_P+_omega_yaw_P)*_G_Dt);
\end{lstlisting}
\noindent\_ins.get\_gyro()为$\omega_{gyro}$,\_omega\_P,\_omega\_yaw\_P和\_omega\_I为$\omega_{correction}$,其中\_omega\_P,\_omega\_yaw\_P为比例项,\_omega\_I为积分项,共同构成PI控制。\_omega\_P和\_omega\_I由函数drift\_correction()计算,\_omega\_yaw\_P由函数drift\_correction\_yaw()计算。

不把比例项\_omega\_P并入\_omega是因为spin\_rate由\_omega.length()计算,而spin\_rate增大会导致\_P\_gain增大,\_P\_gain又被用于计算\_omega\_P和\_omega\_yaw\_P,从而形成正反馈。

%\noindent ============================================
\vspace{10pt}

%222222222222222222222222222222222222222
%222222222222222222222222222222222222222
%222222222222222222222222222222222222222
\noindent2.\textbf{normalize()}

将更新后的DCM矩阵进行标准化处理。
数值计算产生的误差会逐渐积累,破坏DCM矩阵的正交性,因此需要对更新后的DCM矩阵进行处理(orthogonality conditions renormalization)。具体步骤如下:

第一步:计算DCM矩阵行向量X和Y的点积。其值应为0,因此结果可被用来作为误差的度量
\begin{equation}
\mathbf{X}=
\left[ \begin{array}{c}
r_{xx}\\
r_{xy}\\
r_{xz}
\end{array} \right]
\quad
\mathbf{Y}=
\left[ \begin{array}{c}
r_{yx}\\
r_{yy}\\
r_{yz}
\end{array} \right]
\end{equation}
\begin{equation}
error=\mathbf{X}\cdot\mathbf{Y}=\mathbf{X}^T\mathbf{Y}
=[r_{xx}\; r_{xy}\; r_{xz}]
\left[ \begin{array}{c}
r_{yx}\\
r_{yy}\\
r_{yz}
\end{array} \right]
\end{equation}
我们将误差平均分给两个行向量,利用交叉耦合来旋转X和Y:
\begin{equation}
\left[ \begin{array}{c}
r_{xx}\\
r_{xy}\\
r_{xz}
\end{array} \right]_{orthogonal}=
\mathbf{X}_{orthogonal}=\mathbf{X}-\frac{error}{2}\mathbf{Y}
\end{equation}
\begin{equation}
\left[ \begin{array}{c}
r_{yx}\\
r_{yy}\\
r_{yz}
\end{array} \right]_{orthogonal}=
\mathbf{Y}_{orthogonal}=\mathbf{Y}-\frac{error}{2}\mathbf{X}
\end{equation}
这样处理后正交性误差会显著减小。把误差平均分给两个向量与全部分给一个向量相比,能够得到更小的残差。

第二步:调整行向量$\mathbf{Z}$,使其与$\mathbf{X}$和$\mathbf{Y}$正交。只需令$\mathbf{Z}$等于$\mathbf{X}$和$\mathbf{Y}$的叉乘即可:
\begin{equation}
\left[\begin{array}{c}
r_{zx}\\
r_{zy}\\
r_{zz}
\end{array} \right]_{orthogonal}=
\mathbf{Z}_{orthogonal}=\mathbf{X}_{orthogonal}\times\mathbf{Y}_{orthogonal}
\end{equation}

第三步:将DCM矩阵每个行向量单位化。以行向量$\mathbf{X}$为例,
\begin{equation}
\mathbf{X}=\frac{\mathbf{X}}{|\mathbf{x}|}
\end{equation}
此步骤由renorm()函数实现,主要代码为
\begin{lstlisting}
    renorm_val = 1.0f / a.length();
    result = a * renorm_val;    
\end{lstlisting}

\vspace{8pt}
\noindent normalize()函数具体代码:
\begin{lstlisting}
    error = _dcm_matrix.a * _dcm_matrix.b;
    t0 = _dcm_matrix.a - (_dcm_matrix.b * (0.5f * error));
    t1 = _dcm_matrix.b - (_dcm_matrix.a * (0.5f * error));
    t2 = t0 % t1;     // vector cross product

    if (!renorm(t0, _dcm_matrix.a) ||
        !renorm(t1, _dcm_matrix.b) ||
        !renorm(t2, _dcm_matrix.c)) {
        // Our solution is blowing up and we will force back
        // to last euler angles
        reset(true);
    }
\end{lstlisting}

%\noindent ============================================
\vspace{10pt}

%333333333333333333333333333333333333333333333
%333333333333333333333333333333333333333333333
%333333333333333333333333333333333333333333333
\noindent3.\textbf{drift\_correction()}

计算陀螺仪的漂移量,将误差传递给PI控制器得到修正量\_omega\_I,\_omega\_P和\_omega\_yaw\_P,用于反馈补偿到陀螺仪输出的角速率向量。具体分为两个通道:\\
1) Yaw:以磁罗盘或GPS数据为参考,计算陀螺仪偏航角速率的漂移。此部分通过调用函数drift\_correction\_yaw(void)实现,具体分析\hyperref[drift_correction_yaw]{点我}。		\label{drift_correction} %drift_correction_yaw回跳位置
\begin{verbatim}
_omega_yaw_P.z = error_z * _P_gain(spin_rate) * _kp_yaw,
_omega_I_sum.z += error_z * _ki_yaw * yaw_deltat,
\end{verbatim}
2) Roll\&Pitch:以加速度计(或空速管)数据为参考,计算陀螺仪滚转角和俯仰角速率的漂移;
\begin{verbatim}
_omega_P = error * _P_gain(spin_rate) * _kp;
_omega_I_sum += error * _ki * _ra_deltat;
_omega_I += _omega_I_sum;
\end{verbatim}

%两个通道加权求和得到总的修正量:
%\begin{eqnarray}
%&&\mathbf{TotalCorrection}=\\ \nonumber
%&&\qquad W_{RP}\mathbf{RollPitchCorrectionPlane}
%+W_Y\mathbf{YawCorrectionPlane} \nonumber
%\end{eqnarray}
%然后将总的修正量传递给PI控制器:

即:
\begin{eqnarray}
\omega_{P}& = & K_P\cdot error\cdot P_{gain}\\\nonumber
\omega_{yawP}& = & K_{yawP}\cdot error_z\cdot P_{gain}\\\nonumber
\omega_{I}& = & \int_{t_1}^{t_2}{K_I\cdot error\cdot d\tau_{ra}}+\int_{t_1}^{t_2}{K_{yawI}\cdot error_z\cdot d\tau_{yaw}}\\
\omega_{Correction} & = & \omega_{P}+\omega_{yawP}+\omega_{I}\nonumber
\end{eqnarray}
最后将此修正向量补偿到陀螺仪信号,形成反馈控制回路。

Yaw通道的计算是通过调用drift\_correction\_yaw()函数实现的,下面分析Roll\&Pitch通道。
在机体参考系下,加速度计所得加速度向量(绝对加速度)是重力加速度与飞行器相对惯性系的加速度(下面称为地面加速度)的矢量和:
\begin{equation}
\mathbf{A}_b(t)=\mathbf{g}_b(t)+\mathbf{a}_b(t)
\end{equation}
$\mathbf{A}_b(t)$为加速度计的输出,$\mathbf{g}_b(t)$是在机体参考系下的重力加速度,$\mathbf{a}_b(t)$是在机体参考系下的地面加速度。

利用旋转矩阵的估计值将上式转换到惯性系:
\begin{equation}
\hat{\mathbf{R}}(t)\cdot\mathbf{A}_b(t)=\mathbf{g}_e+\mathbf{a}_e(t)
\end{equation}
$\hat{\mathbf{R}}(t)$是由陀螺仪数据得到的旋转矩阵的估计值,$\mathbf{g}_e$和$\mathbf{a}_e(t)$是惯性参考系下的重力加速度和地面加速度。

如果旋转矩阵的估计值是正确的,则上式为一恒等式。但是由于陀螺漂移的存在,实际情况不可能准确满足上式。等号两侧的两个向量的叉乘可用来指示陀螺的漂移量。为了提高精度,将上式积分:
\begin{equation}\label{计算yaw误差的基础}
\int_{t_1}^{t_2}{\hat{\mathbf{R}}_\tau\cdot\mathbf{A}_b(\tau)\cdot d\tau=(t_2-t_1)\cdot\mathbf{g}_e+(\mathbf{V}_e(t_2)-\mathbf{V}_e(t_1))}
\end{equation}
积分后,我们可以从使用加速度信息改为使用速度信息。等号左侧是旋转矩阵和加速度计输出的乘积之和,等号右侧$\mathbf{g}_e$重力加速度是已知的常值向量,第二项是由GPS得到的两个时刻的速度矢量差。将等号两侧两个向量叉乘得到误差旋转向量:
\begin{equation}
\mathbf{error}_{earth}(t_2)=\left(\int_{t_1}^{t_2}{\hat{\mathbf{R}}(\tau)\cdot\mathbf{A}_b(\tau)\cdot d\tau}\right)\times\left((t_2-t_1)\cdot\mathbf{g}_e+(\mathbf{V}_e(t_2)-\mathbf{V}_e(t_1))\right)
\end{equation}
为得到与时间间隔不相关的误差表达式,将上式变形为
\begin{equation}
\mathbf{error}_{earth}(t_2)=\frac{\left[\frac{1}{t_2-t_1}\int_{t_1}^{t_2}{\hat{\mathbf{R}}(\tau)\cdot\mathbf{A}_b(\tau)\cdot d\tau}\right]\times\left[\mathbf{g}_e+\frac{(\mathbf{V}_e(t_2)-\mathbf{V}_e(t_1))}{t_2-t_1}\right]}{\left|\mathbf{g}_e+\frac{(\mathbf{V}_e(t_2)-\mathbf{V}_e(t_1))}{t_2-t_1}\right|}
\end{equation}
式中将加速度参考向量($\mathbf{g}_e+\mathbf{a}_e$)进行了标准化。最后,将误差向量转换到机体参考系中,从而可直接作为PI控制器的输入量:
\begin{equation}
\mathbf{error}_{body}=\hat{\mathbf{R}}^T(t_2)\cdot\mathbf{error}_{earth}(t_2)
\end{equation}

\vspace{8pt}
\noindent 主要代码分析:

1.调用drift\_correction\_yaw()函数计算yaw通道的漂移量以及相应PI控制器修正项:
\begin{lstlisting}
    drift_correction_yaw();
\end{lstlisting}

2.计算式(\ref{计算yaw误差的基础})等号左侧的向量$\int_{t_1}^{t_2}{\hat{\mathbf{R}}_\tau\cdot\mathbf{A}_b(\tau)\cdot d\tau}$:
\begin{lstlisting}
    // rotate accelerometer values into the earth frame
    _accel_ef = _dcm_matrix * _ins.get_accel();
    // integrate the accel vector in the earth frame between GPS readings
    _ra_sum += _accel_ef * deltat;
    // keep a sum of the deltat values, so we know how much time we have integrated over
    _ra_deltat += deltat;
\end{lstlisting}

3.计算式(\ref{计算yaw误差的基础})等号右侧的向量中的速度向量。如果没有GPS或GPS信号差,则使用空速信息来计算速度向量:
\begin{lstlisting}
    if (!have_gps() ||
        _gps->status() < GPS::GPS_OK_FIX_3D ||
        _gps->num_sats < _gps_minsats) {
		...
        velocity = _dcm_matrix.colx() * airspeed;
        velocity += _wind;
\end{lstlisting}
否则,使用GPS信息更新速度向量:
\begin{lstlisting}
        velocity = Vector3f(_gps->velocity_north(), _gps->velocity_east(), _gps->velocity_down());
		...
        // keep last airspeed estimate for dead-reckoning purposes
        Vector3f airspeed = velocity - _wind;
        airspeed.z = 0;
        _last_airspeed = airspeed.length(); 
\end{lstlisting}

4.更新飞行器的位置。如果有GPS,直接使用其数据更新位置:
\begin{lstlisting}
        _last_lat = _gps->latitude;
        _last_lng = _gps->longitude;
        _position_offset_north = 0;
        _position_offset_east = 0;
\end{lstlisting}
否则,使用上一时刻的位置和第3步得到的速度向量进行航迹推算:
\begin{lstlisting}
        _position_offset_north += velocity.x * _ra_deltat;
        _position_offset_east  += velocity.y * _ra_deltat;
\end{lstlisting}

5.计算两个向量的叉乘,得到误差向量,并转换到机体系:
\begin{lstlisting}
    Vector3f GA_e;
    GA_e = Vector3f(0, 0, -1.0f);
	...
    float v_scale = gps_gain.get()/(_ra_deltat*GRAVITY_MSS);
    Vector3f vdelta = (velocity - _last_velocity) * v_scale;
    GA_e += vdelta;
    GA_e.normalize();
    ...
    _ra_sum /= (_ra_deltat * GRAVITY_MSS);

    // get the delayed ra_sum to match the GPS lag
    Vector3f GA_b;
    if (using_gps_corrections) {
        GA_b = ra_delayed(_ra_sum);
    } else {
        GA_b = _ra_sum;
    }
    GA_b.normalize();
    Vector3f error = GA_b % GA_e;
    
    if (use_compass()) {
        if (have_gps() && gps_gain == 1.0f) {
            error.z *= sinf(fabsf(roll));
        } else {
            error.z = 0;
        }
    // convert the error term to body frame
    error = _dcm_matrix.mul_transpose(error);
\end{lstlisting}
其中,GA\_b和GA\_e分别是式(\ref{计算yaw误差的基础})等号左侧和右侧的向量除以(\_ra\_deltat * GRAVITY\_MSS)然后标准化得到的向量。GA\_b为单位加速度向量估计值,GA\_e为单位加速度参考向量。其叉乘即误差向量。注意此通道只需得到滚转和俯仰的漂移量,但叉乘得到的误差向量实际上也包含了偏航的漂移量(error.z),而其已在调用drift\_correction\_yaw()函数计算yaw通道漂移时算过一次并计入了\_omega\_I\_sum.z中,此处如果保留error.z,则下面计算\_omega\_I时会再将其再次计入,从而发生了重复,会产生过大的控制量。因此,21-26行对误差的z轴分量进行处理,只有滚转角较大时(因为此时偏航角误差转换到机体系后在z轴上的分量较小?)使其发生作用。

6.计算\_omega\_P和\_omega\_I。使用constrain\_float()函数对积分项进行限幅是为了防止积分饱和:
\begin{lstlisting}
    // base the P gain on the spin rate
    float spin_rate = _omega.length();
    _omega_P = error * _P_gain(spin_rate) * _kp;
    // accumulate some integrator error
    if (spin_rate < ToRad(SPIN_RATE_LIMIT)) {
        _omega_I_sum += error * _ki * _ra_deltat;
        _omega_I_sum_time += _ra_deltat;
    }
    if (_omega_I_sum_time >= 5) {
        // limit the rate of change of omega_I to the hardware
        // reported maximum gyro drift rate. This ensures that
        // short term errors don't cause a buildup of omega_I
        // beyond the physical limits of the device
        float change_limit = _gyro_drift_limit * _omega_I_sum_time;
        _omega_I_sum.x = constrain_float(_omega_I_sum.x, -change_limit, change_limit);
        _omega_I_sum.y = constrain_float(_omega_I_sum.y, -change_limit, change_limit);
        _omega_I_sum.z = constrain_float(_omega_I_sum.z, -change_limit, change_limit);
        _omega_I += _omega_I_sum;
        _omega_I_sum.zero();
        _omega_I_sum_time = 0;
    }
\end{lstlisting}


%\noindent ============================================
\vspace{10pt}

%4444444444444444444444444444444444444444444444
%4444444444444444444444444444444444444444444444
%4444444444444444444444444444444444444444444444
\noindent4.\textbf{check\_matrix()}

检查DCM矩阵是否有坏值。\\
如果\_dcm\_matrix.is\_nan(),则调用reset()复位;\\
如果$|\_dcm\_matrix.c.x|\ge1$,则调用normalize()进行标准化处理;\\
如果处理后\_dcm\_matrix.is\_nan()或|\_dcm\_matrix.c.x|>10,则调用reset()复位。

%\noindent ============================================
\vspace{10pt}

%5555555555555555555555555555555555555555555555
%5555555555555555555555555555555555555555555555
%5555555555555555555555555555555555555555555555
\noindent5.\textbf{euler\_angles()}

计算欧拉角。\_trim用于补偿传感器与机体坐标系之间的误差(即飞行器静置于地面时由加速度传感器得到的俯仰角和滚转角)。其由AP\_InertialSensor.cpp中的函数\_calculate\_trim()计算。

\vspace{8pt}
\noindent 具体代码:
\begin{lstlisting}
    _body_dcm_matrix = _dcm_matrix;
    _body_dcm_matrix.rotateXYinv(_trim);
    _body_dcm_matrix.to_euler(&roll, &pitch, &yaw);

    roll_sensor     = degrees(roll)  * 100;
    pitch_sensor    = degrees(pitch) * 100;
    yaw_sensor      = degrees(yaw)   * 100;

    if (yaw_sensor < 0)
        yaw_sensor += 36000;
\end{lstlisting}

%==========================================
\vspace{20pt}
\noindent 此cpp中其它重要函数:
\vspace{5pt}

\noindent\textbf{drift\_correction\_yaw(void)}函数:\hyperref[drift_correction]{(返回drift\_correction()函数)} \label{drift_correction_yaw}

使用GPS或磁罗盘数据作为参考向量,来消除陀螺仪在偏航方向的漂移。
磁罗盘所得机头指向(heading)或GPS所得航向向量(COG,course over ground)(仅限无侧滑时,此时航向与机头指向同向)与IMU滚转轴(X)在水平面上的投影向量的夹角可用来衡量陀螺在偏航轴上的漂移量。修正量等于DCM矩阵的列向量X与航向向量的叉乘在Z轴上的分量。
首先得到航向向量:
\begin{equation}
COGX=cos(cog),\quad COGY=sin(cog)
\end{equation}
$cog$为航向角。
然后可得偏航角修正量:
\begin{equation}\label{YawCorrectionGround}
YawCorrectionGround=r_{xx}COGY-r_{yx}COGX
\end{equation}
式(\ref{YawCorrectionGround})得到的修正量是在惯性参考系下的,为修正陀螺漂移,需将其转化到机体参考系。做法是:将其乘以DCM矩阵的行向量Z:
\begin{equation}\label{YawCorrectionPlane}
YawCorrectionPlane=YawCorrectionGround
\left[ \begin{array}{c}
r_{zx}\\
r_{zy}\\
r_{zz}
\end{array}\right]
\end{equation}
最终的修正量只有机体坐标系z轴上的分量($r_{zz}项$)。
%======================================

\vspace{8pt}
\noindent 主要代码分析:

1.如果使用了罗盘,则使用其数据计算陀螺仪在偏航方向的漂移:
\begin{lstlisting}
     if (use_compass()){
            if (!_flags.have_initial_yaw && _compass->read()) {
                float heading = _compass->calculate_heading(_dcm_matrix);
                _dcm_matrix.from_euler(roll, pitch, heading);
                _omega_yaw_P.zero();
                _flags.have_initial_yaw = true;
            }
            new_value = true;
            yaw_error = yaw_error_compass();
     }
\end{lstlisting}
注意代码第3行通过calculate\_heading()函数计算出机头方向与正北的夹角heading,第4行使用heading代替yaw更新了DCM矩阵。即heading被作为yaw的初始值,因为陀螺仪只能积分得到yaw的变化量,而无法给出初始量。
最后调用函数yaw\_error\_compass()计算偏航误差,其主要代码为:
\begin{lstlisting}
    const Vector3f &mag = _compass->get_field();
    Vector2f rb = _dcm_matrix.mulXY(mag);
    rb.normalize();
    ...
    if( _last_declination != _compass->get_declination() ) {
        _last_declination = _compass->get_declination();
        _mag_earth.x = cosf(_last_declination);
        _mag_earth.y = sinf(_last_declination);
    }
    // calculate the error term in earth frame
    // calculate the Z component of the cross product of rb and _mag_earth
    return rb % _mag_earth;
\end{lstlisting}
\noindent 代码中\_last\_declination为地磁偏角,\_mag\_earth为地磁场在水平面内的投影向量。磁罗盘三轴与机体轴一致,因此,正确的DCM乘以罗盘所得向量mag可以得到惯性参考系下的地磁场向量,其与\_mag\_earth平行。如果由陀螺仪得到的yaw是正确的,则第2行中的rb与\_mag\_earth平行,rb叉乘\_mag\_earth等于0;否则,\_dcm\_matrix不能将向量mag转换为惯性参考系下的地磁场向量,此时rb与\_mag\_earth的夹角即陀螺仪所得yaw与真实值的误差。

2.未使用罗盘,如果已假设飞行器沿x轴向前飞行(即无侧滑),且备有GPS,则使用GPS数据计算陀螺仪在偏航方向的漂移:
\begin{lstlisting}
      else if (_flags.fly_forward && have_gps()) { 
        if (_gps->last_fix_time != _gps_last_update &&
            _gps->ground_speed_cm >= GPS_SPEED_MIN) {
            yaw_deltat = (_gps->last_fix_time - _gps_last_update) * 1.0e-3f;
            _gps_last_update = _gps->last_fix_time;
            new_value = true;
            float gps_course_rad = ToRad(_gps->ground_course_cd * 0.01f);
            float yaw_error_rad = wrap_PI(gps_course_rad - yaw);
            yaw_error = sinf(yaw_error_rad);
\end{lstlisting}
在无侧滑的条件下,航向角course即偏航角yaw的真实值,其与由陀螺得到的yaw的差值即误差角度。

当满足以下三种情况之一时,使用GPS得到的航向角重置偏航角yaw:\\
1) 飞行器速度达到了GPS\_SPEED\_MIN且从未得到偏航角信息;\\
2) 上一次从GPS得到偏航角信息已经过20秒,则此时陀螺仪已存在较大漂移;\\
3) 飞行器速度大于3*GPS\_SPEED\_MIN(9m/s),且偏航角误差大于60度。
\begin{lstlisting}
           if (!_flags.have_initial_yaw ||
                yaw_deltat > 20 ||
                (_gps->ground_speed_cm >= 3*GPS_SPEED_MIN && fabsf(yaw_error_rad) >= 1.047f)) {
                // reset DCM matrix based on current yaw
                _dcm_matrix.from_euler(roll, pitch, gps_course_rad);
                _omega_yaw_P.zero();
                _flags.have_initial_yaw = true;
                yaw_error = 0;
\end{lstlisting}

3.计算偏航角速度的比例修正项\_omega\_yaw\_P和\_omega\_I\_sum.z(用于计算角速度的积分修正项omega\_I的z轴分量),公式在第5和11行:
\begin{lstlisting}
    //将误差向量由惯性系转换到机体系(公式23)
    float error_z = _dcm_matrix.c.z * yaw_error;

    float spin_rate = _omega.length();
    _omega_yaw_P.z = error_z * _P_gain(spin_rate) * _kp_yaw;
    if (_flags.fast_ground_gains) {
        _omega_yaw_P.z *= 8;
    }

    if (yaw_deltat < 2.0f && spin_rate < ToRad(SPIN_RATE_LIMIT)) {
        _omega_I_sum.z += error_z * _ki_yaw * yaw_deltat;
    }
    _error_yaw_sum += fabsf(yaw_error);
    _error_yaw_count++;
\end{lstlisting}


\vspace{10pt}
\noindent\textbf{ra\_delayed}()函数:

由于GPS存在延迟(AHRS\_GPS\_DELAY),在drift\_correction()函数中求俯仰滚转通道的漂移时对两个加速度向量进行叉乘,这两个向量分别用到GPS的数据和加速度计的数据。为使数据同步,需给直接由加速度计得到的加速度施加相应的延迟。此函数返回经过延迟的加速度向量。


%==================================================
\vspace{30pt}
\noindent 参考资料:\\
1.Direction Cosine Matrix IMU: Theory, William Premerlani and Paul Bizard\\
2.Roll-Pitch Gyro Drift Compensation, William Premerlani.\\
http://gentlenav.googlecode.com/files/RollPitchDriftCompensation.pdf


\begin{flushright}
\vspace{30pt}
2015.7.29\\
方酉
\end{flushright}
\end{document}